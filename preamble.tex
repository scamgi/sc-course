% File: preamble.tex
% --- Codifica e Lingua ---
\usepackage[utf8]{inputenc}
\usepackage[T1]{fontenc}
\usepackage[italian]{babel}

% --- Gestione della pagina ---
\usepackage[a4paper, margin=2.5cm]{geometry}

% --- Pacchetti per la Matematica ---
\usepackage{amsmath}   % Funzionalità matematiche avanzate
\usepackage{amssymb}   % Simboli matematici
\usepackage{amsfonts}  % Font matematici

% --- Grafica e Colori ---
\usepackage{graphicx}  % Per includere immagini
\usepackage{xcolor}    % Per usare i colori

% --- Intestazioni e Piè di pagina ---
\usepackage{fancyhdr}
\pagestyle{fancy}
\setlength{\headheight}{14pt}
\fancyhf{} % Pulisce tutte le intestazioni e i piè di pagina
\fancyhead[L]{Corso di Preparazione - Segnali per le Comunicazioni}
\fancyfoot[C]{\thepage}
\renewcommand{\headrulewidth}{0.4pt}
\renewcommand{\footrulewidth}{0.4pt}

% --- Pacchetti per la strutturazione del documento ---
\usepackage{pgffor} 
\usepackage{pgfplots}
\pgfplotsset{compat=1.18}

% --- Collegamenti ipertestuali ---
\usepackage{hyperref}
\hypersetup{
    colorlinks=true,
    linkcolor=blue,
    filecolor=magenta,      
    urlcolor=cyan,
    pdftitle={Eserciziario di Segnali},
    pdfpagemode=FullScreen,
}

% --- Definizioni di ambienti personalizzati ---
\usepackage{amsthm}
\usepackage{framed}

% Stile per le soluzioni (amsthm works fine with framed here)
\theoremstyle{plain}
\newtheorem{soluzioneinner}{Soluzione Esercizio}

% Manual definition for 'esercizio' to avoid amsthm/framed conflict
\newcounter{esercizio}
\newenvironment{esercizio}[1]%
{%
    \refstepcounter{esercizio}%
    \begin{leftbar}%
    \noindent\textbf{Esercizio \theesercizio: #1}\par\nobreak\vspace{\topsep}%
}%
{%
    \end{leftbar}%
}

% Ambiente riquadrato per le soluzioni
\newenvironment{soluzione}[1]
  {\begin{soluzioneinner}[#1]\begin{framed}}
  {\end{framed}\end{soluzioneinner}}