% File: problems/ex18.tex

\begin{esercizio}{Prodotti di Segnali \textit{x(t)}}
    Un segnale tempo-continuo è definito dal prodotto di due funzioni seno cardinale:
    \[
        x(t) = \text{sinc}(10t) \cdot \text{sinc}(20t)
    \]
    \begin{enumerate}
        \item Calcolare la sua trasformata di Fourier $X(f)$.
        (Suggerimento: il prodotto nel tempo corrisponde a una convoluzione in frequenza).
        \item Qual è la banda del segnale $x(t)$?
        \item Determinare la minima frequenza di campionamento $f_s$ necessaria per campionare $x(t)$ senza aliasing.
    \end{enumerate}
\end{esercizio}