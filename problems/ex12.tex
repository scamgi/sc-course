% File: problems/ex12.tex

\begin{esercizio}{La DFT di una Sinusoide Campionata}
    Una sequenza tempo-discreto di $N=16$ campioni è definita da:
    \[
        x[n] = 4 \cos\left(2\pi \frac{3}{16} n\right) \quad \text{per } 0 \leq n \leq 15
    \]
    \begin{enumerate}
        \item Ricordando l'identità di Eulero $\cos(\theta) = \frac{1}{2}(e^{j\theta} + e^{-j\theta})$.
        \item Sfruttare la proprietà di "shifting" della DFT per scrivere l'espressione della sua DFT a 16 punti, $X_k$, senza calcolare esplicitamente la sommatoria. 
    \end{enumerate}
    (Suggerimento: la DFT di una sinusoide campionata in questo modo è composta da soli due impulsi).
\end{esercizio}