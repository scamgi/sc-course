% File: problems/ex9.tex

\begin{esercizio}{Calcolare il Segnale Ricostruito \textit{x\textsubscript{R}(t)} con Aliasing}
    Facendo riferimento al segnale e alla frequenza di campionamento dell'Esercizio 8 ($x(t) = \cos(2\pi \cdot 8t)$, campionato a $f_s = 10$ Hz), si determini l'espressione analitica completa del segnale ricostruito nel tempo, $x_R(t)$.
    
    Ricordare che il processo di campionamento ideale introduce un fattore di scala $1/T$ nell'ampiezza dello spettro.
    
    Il segnale ricostruito $x_R(t)$ sarà ancora una cosinusoide, ma con frequenza e ampiezza diverse rispetto al segnale originale.
\end{esercizio}
