% File: problems/ex17.tex

\begin{esercizio}{Analisi Grafica in Frequenza Normalizzata}
    La trasformata di Fourier di una sequenza tempo-discreto $x[n]$ è data da:
    \[
        \tilde{X}(\nu) = \sum_{k=-\infty}^{\infty} \left( \delta(\nu - 0.4 - k) + \delta(\nu + 0.4 - k) \right)
    \]
    Questa sequenza è stata ottenuta campionando un segnale tempo-continuo $x(t) = 2\cos(2\pi f_0 t)$ con una frequenza di campionamento $f_s=50$ Hz.
    
    \begin{enumerate}
        \item Qual era la frequenza $f_0$ del segnale originale?
        \item Spiegare se il fenomeno dell'aliasing è avvenuto e perché.
        \item Quale segnale $x_R(t)$ si otterrebbe applicando un filtro passa-basso ideale con banda da $-f_s/2$ a $f_s/2$?
    \end{enumerate}
\end{esercizio}