% File: problems/ex25.tex

\begin{esercizio}{Autovalutazione e Domande Concettuali}
    Rispondere alle seguenti domande motivando brevemente la risposta.
    
    \begin{enumerate}
        \item Un segnale $x(t)$ con banda massima $B=5$ kHz viene campionato con $f_s = 8$ kHz. Il segnale ricostruito $x_R(t)$ sarà identico a quello originale? Perché?
        
        \item Si ha una sequenza $x[n] = \cos(2\pi \frac{n}{5})$. Quanti campioni $N$ deve contenere al minimo la sequenza affinché la sua DFT a $N$ punti contenga esattamente due impulsi non nulli (escludendo le periodicità della DFT)?
        
        \item Dopo un campionamento e una ricostruzione, lo spettro di una sinusoide appare centrato a $f_{apparente} = 1$ kHz. Se la frequenza di campionamento era $f_s = 10$ kHz, quali erano le due più basse frequenze possibili, $f_{originale} > 0$, del segnale di partenza?
        
        \item Se la DFT di $x[n]$ è $X_k$, qual è la DFT della sequenza $y[n]=x[n](-1)^n$? 
        (Suggerimento: $(-1)^n = e^{j\pi n} = e^{j2\pi \frac{N/2}{N} n}$).
    \end{enumerate}
\end{esercizio}