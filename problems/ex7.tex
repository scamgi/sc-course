% File: problems/ex7.tex

\begin{esercizio}{Calcolare lo Spettro Ricostruito con Aliasing}
    Partendo dallo scenario di aliasing descritto nell'Esercizio 6:
    
    \begin{enumerate}
        \item Si applichi un filtro passa-basso ideale $H_R(f)$ con frequenza di taglio $f_c = f_s/2 = 75$ Hz e guadagno unitario.
        \item Determinare l'espressione analitica e disegnare il grafico dello spettro ricostruito $X_R(f)$ nella banda $|f| \leq 75$ Hz. Lo spettro risultante non sarà più un semplice triangolo.
    \end{enumerate}
    
    (Suggerimento: lo spettro ricostruito è la somma, nella banda base, della coda della replica centrale e della coda della replica adiacente).
\end{esercizio}