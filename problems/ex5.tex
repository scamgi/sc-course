% File: problems/ex5.tex

\begin{esercizio}{Ricostruzione Ideale del Segnale}
    Considerando il processo di campionamento descritto nell'Esercizio 4 (segnale con spettro triangolare campionato a $f_s = 300$ Hz).
    
    \begin{enumerate}
        \item Descrivere le caratteristiche del filtro passa-basso ideale $H_R(f)$ necessario per ricostruire perfettamente il segnale originale $x(t)$ a partire dallo spettro campionato $\tilde{X}(f)$. Specificare la sua frequenza di taglio e il suo guadagno.
        \item Scrivere l'espressione matematica dello spettro ricostruito $X_R(f) = \tilde{X}(f) \cdot H_R(f)$.
        \item Qual è l'espressione del segnale ricostruito nel tempo, $x_R(t)$?
    \end{enumerate}
\end{esercizio}