% File: problems/ex6.tex

\begin{esercizio}{Visualizzare l'Aliasing}
    Si consideri nuovamente il segnale $x(t)$ dell'Esercizio 4, con una trasformata di Fourier a forma di triangolo, nulla per $|f| > 100$ Hz:
    \[
        X(f) = 
        \begin{cases} 
            1 - \frac{|f|}{100} & \text{se } |f| \leq 100 \text{ Hz} \\
            0 & \text{altrimenti}
        \end{cases}
    \]
    Questa volta, il segnale viene campionato con una frequenza di campionamento $f_s = 150$ Hz.
    
    \begin{enumerate}
        \item Spiegare perché questa frequenza di campionamento introduce aliasing.
        \item Disegnare un grafico qualitativo dello spettro del segnale campionato, $\tilde{X}(f)$, nell'intervallo di frequenze da $-200$ Hz a $200$ Hz, mostrando chiaramente la sovrapposizione tra le repliche spettrali.
    \end{enumerate}
\end{esercizio}