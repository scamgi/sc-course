% File: problems/ex15.tex

\begin{esercizio}{Flusso Completo: da \textit{x(t)} alla DFT}
    Si consideri il segnale tempo-continuo $x(t) = \cos(8\pi t)$.
    Il segnale viene campionato con una frequenza $f_s = 20$ Hz, ottenendo la sequenza $x[n]$. Si consideri un blocco di $N=10$ campioni per il calcolo della DFT.
    
    \begin{enumerate}
        \item Determinare l'intervallo di campionamento $T$.
        \item Scrivere l'espressione per la sequenza discreta $x[n] = x(nT)$.
        \item Riscrivere $x[n]$ nella forma $A \cos(2\pi \frac{k}{N}n)$ per identificare i valori di $A$, $k$ ed $N$.
        \item Scrivere infine l'espressione della DFT a 10 punti, $X_k$.
    \end{enumerate}
\end{esercizio}