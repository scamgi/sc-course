% File: solutions/ex19.tex

\begin{soluzione}{19}
    Questo problema si risolve in modo efficiente lavorando interamente nel dominio della frequenza e applicando la linearità della Trasformata di Fourier Inversa (IDFT).

    \subsubsection*{1. Calcolare la DFT originale, $X_k$}
    Per prima cosa, determiniamo la DFT della sequenza $x[n]$. La sequenza è nella forma standard $A\cos(2\pi \frac{k_0}{N}n)$ con:
    \begin{itemize}
        \item $A = 2$
        \item $N = 8$
        \item $k_0 = 2$
    \end{itemize}
    La sua DFT è data da due impulsi:
    \[
        X_k = \frac{AN}{2} \left( \delta[k-k_0] + \delta[k-(N-k_0)] \right)
    \]
    Calcoliamo l'ampiezza degli impulsi: $\frac{AN}{2} = \frac{2 \cdot 8}{2} = 8$.
    
    L'indice coniugato è $N-k_0 = 8-2 = 6$.
    
    Quindi, la DFT della sequenza originale è:
    \[
        X_k = 8 \cdot \delta[k-2] + 8 \cdot \delta[k-6]
    \]
    Questo ci dice che l'unico campione non nullo che ci interessa è $X_2 = 8$.

    \subsubsection*{2. Determinare la DFT modificata, $Y_k$}
    La nuova DFT, $Y_k$, si ottiene sottraendo il termine $X_2 \cdot \delta[k-2]$ da $X_k$.
    \[
        Y_k = X_k - X_2 \cdot \delta[k-2]
    \]
    Sostituiamo le espressioni che abbiamo trovato:
    \begin{align*}
        Y_k &= \left( 8 \cdot \delta[k-2] + 8 \cdot \delta[k-6] \right) - (8) \cdot \delta[k-2] \\
        &= 8 \cdot \delta[k-2] - 8 \cdot \delta[k-2] + 8 \cdot \delta[k-6] \\
        &= 8 \cdot \delta[k-6]
    \end{align*}
    La DFT modificata, $Y_k$, consiste in un singolo impulso all'indice $k=6$ con ampiezza 8.

    \subsubsection*{3. Calcolare la Sequenza nel Tempo, $y[n]$}
    Ora dobbiamo trovare la sequenza $y[n]$ la cui DFT è $Y_k = 8 \cdot \delta[k-6]$. Usiamo la Trasformata di Fourier Inversa (IDFT).
    
    Ricordiamo la coppia notevole per la IDFT di un impulso:
    \[
        \mathcal{F}^{-1}\left\{ C \cdot \delta[k-k_0] \right\} = \frac{C}{N} e^{j2\pi\frac{k_0 n}{N}}
    \]
    Nel nostro caso:
    \begin{itemize}
        \item $C = 8$
        \item $k_0 = 6$
        \item $N = 8$
    \end{itemize}
    Sostituendo questi valori, otteniamo l'espressione per $y[n]$:
    \[
        y[n] = \frac{8}{8} e^{j2\pi\frac{6n}{8}}
    \]
    Semplificando:
    \[
        \mathbf{y[n] = e^{j\frac{3\pi}{2}n}}
    \]
    Questa è l'espressione finale per la sequenza nel tempo. A differenza della sequenza originale, che era reale (un coseno), la sequenza risultante è complessa, poiché il suo spettro ($Y_k$) non possiede la simmetria Hermitiana (cioè non ha un impulso coniugato in $k=N-6=2$).
    
\end{soluzione}