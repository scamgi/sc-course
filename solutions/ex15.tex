% File: solutions/ex15.tex

\begin{soluzione}{15}
    Questo esercizio ci guida attraverso il processo completo che parte da un segnale tempo-continuo e arriva alla sua DFT, evidenziando le relazioni tra le frequenze e i parametri di campionamento.

    \begin{enumerate}
        \item \textbf{Intervallo di campionamento $T$}
        
        L'intervallo di campionamento $T$ è semplicemente l'inverso della frequenza di campionamento $f_s$.
        \[
            T = \frac{1}{f_s} = \frac{1}{20 \text{ Hz}} = \mathbf{0.05 \text{ s}}
        \]
        
        \item \textbf{Espressione della sequenza discreta $x[n]$}
        
        La sequenza si ottiene sostituendo $t = nT$ nell'espressione del segnale originale $x(t) = \cos(8\pi t)$.
        \[
            x[n] = x(nT) = \cos(8\pi \cdot nT) = \cos(8\pi \cdot n \cdot 0.05)
        \]
        Semplificando l'argomento del coseno:
        \[
            x[n] = \cos(8\pi n \cdot \frac{1}{20}) = \cos\left(\frac{8\pi n}{20}\right) = \cos\left(\frac{2\pi n}{5}\right)
        \]
        L'espressione della sequenza campionata è quindi:
        \[
            x[n] = \cos\left(\frac{2\pi}{5}n\right)
        \]
        
        \item \textbf{Riscrivere $x[n]$ nella forma standard per la DFT}
        
        L'obiettivo è far corrispondere l'espressione trovata, $x[n] = \cos(\frac{2\pi}{5}n)$, con la forma standard $A \cos(2\pi \frac{k}{N}n)$. Questo ci permetterà di identificare direttamente la "frequenza discreta" $k$ per la DFT.
        
        I dati per la DFT sono:
        \begin{itemize}
            \item $N = 10$ (lunghezza della DFT).
            \item $A = 1$ (ampiezza del coseno).
        \end{itemize}
        Dobbiamo trovare un intero $k$ tale che $\frac{k}{N}$ sia uguale al coefficiente di $2\pi n$ nell'argomento del nostro coseno.
        \[
            \frac{k}{N} = \frac{1}{5} \implies \frac{k}{10} = \frac{1}{5}
        \]
        Risolvendo per $k$:
        \[
            k = \frac{10}{5} = 2
        \]
        Il valore di $k$ è 2. La sequenza può quindi essere riscritta esattamente nella forma desiderata:
        \[
            \mathbf{x[n] = \cos\left(2\pi \frac{2}{10}n\right)}
        \]
        Questo ci dice che la sequenza completa un numero intero di cicli (esattamente 2) all'interno della finestra di osservazione di 10 campioni. Questo previene la dispersione spettrale (spectral leakage).
        
        \item \textbf{Espressione della DFT a 10 punti}
        
        Ora che abbiamo la sequenza nella forma standard, possiamo applicare direttamente la proprietà della DFT per una cosinusoide, come fatto nell'Esercizio 12.
        
        La DFT di $A\cos(2\pi \frac{k_0}{N}n)$ è $\frac{AN}{2}(\delta[k-k_0] + \delta[k-(N-k_0)])$.
        
        Con i nostri parametri:
        \begin{itemize}
            \item $A=1$
            \item $N=10$
            \item $k_0=2$
            \item $N-k_0 = 10-2 = 8$
        \end{itemize}
        Calcoliamo l'ampiezza degli impulsi della DFT:
        \[
            \frac{AN}{2} = \frac{1 \cdot 10}{2} = 5
        \]
        L'espressione della DFT è quindi:
        \[
            \mathbf{X_k = 5 \left( \delta[k - 2] + \delta[k - 8] \right)}
        \]
        Questo significa che la DFT $X_k$ è nulla ovunque, tranne che per $X_2 = 5$ e $X_8 = 5$.
    \end{enumerate}
\end{soluzione}