% File: solutions/ex18.tex

\begin{soluzione}{18}
    \begin{enumerate}
        \item \textbf{Trasformata di Fourier di $x(t)$}
        
        Il segnale $x(t)$ è il prodotto di due segnali nel dominio del tempo:
        \[
            x(t) = g(t) \cdot h(t)
        \]
        dove $g(t) = \text{sinc}(10t)$ e $h(t) = \text{sinc}(20t)$.
        
        La proprietà di moltiplicazione della Trasformata di Fourier afferma che la trasformata di un prodotto nel tempo è la convoluzione delle trasformate in frequenza:
        \[
            X(f) = \mathcal{F}\{g(t) \cdot h(t)\} = G(f) * H(f)
        \]
        Per prima cosa, troviamo le trasformate $G(f)$ e $H(f)$ usando la coppia notevole $\mathcal{F}\{A \cdot \text{sinc}(2Bt)\} = \frac{A}{2B} \text{rect}(\frac{f}{2B})$.
        \begin{itemize}
            \item Per $g(t) = \text{sinc}(10t)$, abbiamo $A=1$ e $2B=10 \implies B=5$.
            La sua trasformata è $G(f) = \frac{1}{10} \text{rect}\left(\frac{f}{10}\right)$. Questo è un rettangolo di altezza $1/10$ che si estende da $-5$ Hz a $5$ Hz.
            
            \item Per $h(t) = \text{sinc}(20t)$, abbiamo $A=1$ e $2B=20 \implies B=10$.
            La sua trasformata è $H(f) = \frac{1}{20} \text{rect}\left(\frac{f}{20}\right)$. Questo è un rettangolo di altezza $1/20$ che si estende da $-10$ Hz a $10$ Hz.
        \end{itemize}
        
        Ora dobbiamo calcolare la convoluzione tra questi due rettangoli:
        \[
            \mathbf{X(f) = \left[ \frac{1}{10} \text{rect}\left(\frac{f}{10}\right) \right] * \left[ \frac{1}{20} \text{rect}\left(\frac{f}{20}\right) \right]}
        \]
        Il risultato della convoluzione di due rettangoli centrati nell'origine è una funzione a forma di trapezio.
        
        \item \textbf{Banda del segnale $x(t)$}
        
        La banda di un segnale risultante dalla convoluzione di due segnali, $X(f) = G(f) * H(f)$, è la \textbf{somma} delle bande dei segnali originali.
        
        Siano $B_g$ e $B_h$ le massime frequenze dei segnali $G(f)$ e $H(f)$ rispettivamente:
        \begin{itemize}
            \item $B_g = 5$ Hz
            \item $B_h = 10$ Hz
        \end{itemize}
        La massima frequenza (la banda) del segnale risultante $X(f)$ è:
        \[
            B_x = B_g + B_h = 5 \text{ Hz} + 10 \text{ Hz} = \mathbf{15 \text{ Hz}}
        \]
        Lo spettro $X(f)$ a forma di trapezio sarà quindi non nullo nell'intervallo $[-15, 15]$ Hz.
        
        \item \textbf{Minima frequenza di campionamento}
        
        Per campionare il segnale $x(t)$ senza introdurre aliasing, dobbiamo rispettare la condizione del Teorema di Nyquist:
        \[
            f_s \ge 2 B_x
        \]
        Dove $B_x$ è la massima frequenza del segnale $x(t)$ che abbiamo appena calcolato.
        
        Sostituendo il valore di $B_x$:
        \[
            f_s \ge 2 \cdot 15 \text{ Hz}
        \]
        La minima frequenza di campionamento è quindi:
        \[
            f_{s, \text{min}} = \mathbf{30 \text{ Hz}}
        \]
        
    \end{enumerate}
\end{soluzione}