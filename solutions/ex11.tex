% File: solutions/ex11.tex

\begin{soluzione}{11}
    L'obiettivo è calcolare la Trasformata di Fourier Discreta (DFT) a $N=4$ punti della sequenza $x[n] = \{1, 0, -1, 0\}$.
    
    \subsubsection*{1. Impostazione della Formula}
    La formula generale della DFT è:
    \[
        X_k = \sum_{n=0}^{N-1} x[n] e^{-j2\pi\frac{kn}{N}}
    \]
    Sostituendo $N=4$, la formula per questo specifico caso diventa:
    \[
        X_k = \sum_{n=0}^{3} x[n] e^{-j2\pi\frac{kn}{4}} = \sum_{n=0}^{3} x[n] e^{-j\frac{\pi}{2}kn}
    \]
    Poiché $x[1]$ e $x[3]$ sono nulli, la sommatoria si semplifica notevolmente. Espandendo la somma:
    \begin{align*}
        X_k &= x[0] \cdot e^{-j\frac{\pi}{2}k(0)} + x[1] \cdot e^{-j\frac{\pi}{2}k(1)} + x[2] \cdot e^{-j\frac{\pi}{2}k(2)} + x[3] \cdot e^{-j\frac{\pi}{2}k(3)} \\
        &= (1) \cdot e^{0} + (0) \cdot e^{-j\frac{\pi}{2}k} + (-1) \cdot e^{-j\pi k} + (0) \cdot e^{-j\frac{3\pi}{2}k} \\
        &= 1 - e^{-j\pi k}
    \end{align*}
    Useremo questa formula semplificata per calcolare i quattro valori di $X_k$.

    \subsubsection*{2. Calcolo dei Componenti della DFT}
    
    \textbf{Per k = 0 (Componente DC):}
    \[
        X_0 = 1 - e^{-j\pi(0)} = 1 - e^{0} = 1 - 1 = \mathbf{0}
    \]
    (Questo corrisponde alla somma dei campioni: $1+0-1+0 = 0$).
    
    \textbf{Per k = 1:}
    \[
        X_1 = 1 - e^{-j\pi(1)} = 1 - e^{-j\pi}
    \]
    Ricordando che $e^{-j\pi} = \cos(-\pi) + j\sin(-\pi) = -1$.
    \[
        X_1 = 1 - (-1) = \mathbf{2}
    \]
    
    \textbf{Per k = 2:}
    \[
        X_2 = 1 - e^{-j\pi(2)} = 1 - e^{-j2\pi}
    \]
    Ricordando che $e^{-j2\pi} = \cos(-2\pi) + j\sin(-2\pi) = 1$.
    \[
        X_2 = 1 - 1 = \mathbf{0}
    \]
    
    \textbf{Per k = 3:}
    \[
        X_3 = 1 - e^{-j\pi(3)} = 1 - e^{-j3\pi}
    \]
    Ricordando che $e^{-j3\pi} = \cos(-3\pi) + j\sin(-3\pi) = -1$.
    \[
        X_3 = 1 - (-1) = \mathbf{2}
    \]
    (Nota: per una sequenza reale, $X_{N-k} = X_k^*$. Qui $X_3 = X_{4-1} = X_1^*$. Poiché $X_1=2$ è reale, $X_1^*=2$, il che conferma il nostro calcolo).
    
    \subsubsection*{3. Risultato Finale}
    La DFT della sequenza data è:
    \[
        \mathbf{X_k = \{0, 2, 0, 2\}}
    \]

\end{soluzione}