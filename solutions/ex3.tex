% File: solutions/ex3.tex

\begin{soluzione}{3}
    Per risolvere questo esercizio è necessario applicare il \textbf{Teorema del Campionamento di Nyquist-Shannon}. Questo teorema stabilisce la condizione minima per campionare un segnale a banda limitata in modo da poterlo ricostruire perfettamente, senza perdita di informazione (aliasing).

    Dal testo del problema, sappiamo che il segnale $x(t)$ ha una banda limitata. La sua massima componente in frequenza, che indichiamo con $B$, è:
    \[
        B = 150 \text{ Hz}
    \]

    \begin{enumerate}
        \item \textbf{Minima frequenza di campionamento ($f_{s, \text{min}}$)}
        
        Il teorema afferma che la frequenza di campionamento $f_s$ deve essere almeno il doppio della massima frequenza del segnale. Matematicamente, la condizione per evitare aliasing è:
        \[
            f_s \ge 2B
        \]
        La minima frequenza di campionamento possibile, nota come \textit{frequenza di Nyquist}, si ha quando vale l'uguaglianza:
        \[
            f_{s, \text{min}} = 2B
        \]
        Sostituendo il valore di $B$:
        \[
            f_{s, \text{min}} = 2 \cdot 150 \text{ Hz} = \mathbf{300 \text{ Hz}}
        \]
        
        \item \textbf{Massimo intervallo di campionamento ($T_{\text{max}}$)}
        
        L'intervallo di campionamento $T$ è l'inverso della frequenza di campionamento $f_s$:
        \[
            T = \frac{1}{f_s}
        \]
        Per trovare l'intervallo di campionamento \textit{massimo} ($T_{\text{max}}$) che ancora evita l'aliasing, dobbiamo usare la frequenza di campionamento \textit{minima} ($f_{s, \text{min}}$) calcolata in precedenza. Un intervallo più lungo (corrispondente a una frequenza più bassa) causerebbe aliasing.
        
        Quindi:
        \[
            T_{\text{max}} = \frac{1}{f_{s, \text{min}}} = \frac{1}{300 \text{ Hz}} = \mathbf{\frac{1}{300} \text{ s}}
        \]
        Che è circa $3.33$ millisecondi.

    \end{enumerate}
\end{soluzione}