% File: solutions/ex1.tex

\begin{soluzione}{1}
    \begin{enumerate}
        \item \textbf{Espressione analitica della sequenza:}
        
        Per ottenere l'espressione della sequenza tempo-discreto $x[n]$ a partire dal segnale tempo-continuo $x(t)$, si deve valutare $x(t)$ negli istanti di campionamento $t = nT$.
        
        La relazione fondamentale è:
        \[
            x[n] = x(nT)
        \]
        Con i dati del problema, abbiamo $x(t) = 4t - 2$ e $T = 0.25$ s. Sostituendo:
        \[
            x[n] = 4(n \cdot T) - 2 = 4(n \cdot 0.25) - 2
        \]
        Semplificando il prodotto $4 \cdot 0.25 = 1$, si ottiene l'espressione finale per la sequenza:
        \[
            \mathbf{x[n] = n - 2}
        \]
        
        \item \textbf{Calcolo dei primi 5 campioni:}
        
        Ora utilizziamo l'espressione appena trovata, $x[n] = n - 2$, per calcolare i valori della sequenza per gli indici $n = 0, 1, 2, 3, 4$.
        \begin{itemize}
            \item Per $n=0: \quad x[0] = 0 - 2 = \mathbf{-2}$
            \item Per $n=1: \quad x[1] = 1 - 2 = \mathbf{-1}$
            \item Per $n=2: \quad x[2] = 2 - 2 = \mathbf{0}$
            \item Per $n=3: \quad x[3] = 3 - 2 = \mathbf{1}$
            \item Per $n=4: \quad x[4] = 4 - 2 = \mathbf{2}$
        \end{itemize}
        La sequenza dei primi campioni è quindi $\{-2, -1, 0, 1, 2, \dots\}$.
    \end{enumerate}
\end{soluzione}