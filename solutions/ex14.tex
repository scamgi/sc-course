% File: solutions/ex14.tex

\begin{soluzione}{14}
    \begin{enumerate}
        \item \textbf{Espressione della sequenza discreta $x[n]$}
        
        La sequenza discreta si ottiene campionando il segnale continuo $x(t)$ a intervalli $t=nT$. I dati del problema sono:
        \begin{itemize}
            \item $x(t) = 5 \cdot \text{sinc}(t - 3)$
            \item $T = 1$ s
        \end{itemize}
        L'espressione per $x[n]$ è quindi:
        \[
            x[n] = x(n T) = x(n \cdot 1) = x(n) = 5 \cdot \text{sinc}(n - 3)
        \]
        La funzione seno cardinale, $\text{sinc}(z)$, ha una proprietà fondamentale: è uguale a 1 quando il suo argomento $z=0$, ed è uguale a 0 per tutti gli altri valori interi di $z$.
        \[
            \text{sinc}(z) = 
            \begin{cases} 
                1 & \text{se } z = 0 \\
                0 & \text{se } z \text{ è un intero non nullo}
            \end{cases}
        \]
        Nel nostro caso, l'argomento è $(n-3)$. Questo argomento è zero solo quando $n=3$, e un intero non nullo per tutti gli altri valori interi di $n$.
        
        Di conseguenza, la sequenza $x[n]$ è non nulla solo per $n=3$:
        \begin{itemize}
            \item Se $n = 3$, allora $x[3] = 5 \cdot \text{sinc}(3-3) = 5 \cdot \text{sinc}(0) = 5 \cdot 1 = 5$.
            \item Se $n \neq 3$ (con n intero), allora $x[n] = 5 \cdot \text{sinc}(\text{intero} \neq 0) = 5 \cdot 0 = 0$.
        \end{itemize}
        La sequenza può quindi essere scritta usando la funzione \textbf{delta di Kronecker}, $\delta[m]$:
        \[
            \mathbf{x[n] = 5 \cdot \delta[n - 3]}
        \]

        \item \textbf{DFT a $N=10$ punti di $x[n]$}
        
        Dobbiamo ora calcolare la DFT della sequenza $x[n] = 5\delta[n-3]$ su un blocco di $N=10$ campioni.
        
        Ricordiamo la coppia fondamentale della DFT per un delta traslato:
        \[
            \mathcal{F}\left\{ A \cdot \delta[n - n_0] \right\} = A \cdot e^{-j2\pi\frac{k n_0}{N}}
        \]
        Questa proprietà ci dice che la DFT di un impulso traslato è un esponenziale complesso.
        
        Nel nostro caso specifico:
        \begin{itemize}
            \item Ampiezza: $A = 5$
            \item Traslazione: $n_0 = 3$
            \item Lunghezza della DFT: $N = 10$
        \end{itemize}
        Sostituendo questi valori nella formula:
        \[
            X_k = 5 \cdot e^{-j2\pi\frac{k \cdot 3}{10}}
        \]
        Semplificando l'esponente:
        \[
            \mathbf{X_k = 5 \cdot e^{-j\frac{3\pi}{5}k}} \quad \text{per } k = 0, 1, \dots, 9
        \]
        Questa è l'espressione analitica della DFT. A differenza degli esercizi precedenti, qui la DFT non è sparsa (cioè non ha zeri), ma è una sequenza di 10 valori complessi il cui modulo è costante e pari a 5, e la cui fase varia linearmente con $k$.
    \end{enumerate}
\end{soluzione}